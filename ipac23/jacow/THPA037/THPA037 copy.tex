\documentclass[letterpaper,
               %boxit,        % check whether paper is inside correct margins
               %titlepage,    % separate title page
               %refpage       % separate references
               %biblatex,     % biblatex is used
               nospread,     % flushend option: do not fill with whitespace to balance columns
               %hyphens,      % allow \url to hyphenate at "-" (hyphens)
               %xetex,        % use XeLaTeX to process the file
               %luatex,       % use LuaLaTeX to process the file
               ]{jacow}
%
% ONLY FOR \footnote in table/tabular
%
%===\usepackage{pdfpages,multirow,ragged2e} %
%
% CHANGE SEQUENCE OF GRAPHICS EXTENSION TO BE EMBEDDED
% ----------------------------------------------------
% test for XeTeX where the sequence is by default eps-> pdf, jpg, png, pdf, ...
%    and the JACoW template provides JACpic2v3.eps and JACpic2v3.jpg which
%    might generates errors, therefore PNG and JPG first
%
\makeatletter%
	\ifboolexpr{bool{xetex}}
	 {\renewcommand{\Gin@extensions}{.pdf,%
	                    .png,.jpg,.bmp,.pict,.tif,.psd,.mac,.sga,.tga,.gif,%
	                    .eps,.ps,%
	                    }}{}
\makeatother

% CHECK FOR XeTeX/LuaTeX BEFORE DEFINING AN INPUT ENCODING
% --------------------------------------------------------
%   utf8  is default for XeTeX/LuaTeX
%   utf8  in LaTeX only realises a small portion of codes
%
\ifboolexpr{bool{xetex} or bool{luatex}} % test for XeTeX/LuaTeX
 {}                                      % input encoding is utf8 by default
 {\usepackage[utf8]{inputenc}}           % switch to utf8

\usepackage[USenglish]{babel}

\ifboolexpr{bool{jacowbiblatex}}%
 {
  \addbibresource{jacow-test.bib}
  \addbibresource{biblatex-examples.bib}
 }{}
\listfiles


\begin{document}
\title{Stability analysis of double-harmonic cavity system in heavy beam
   loading with its feedback loops by a mathematical method based on Pedersen model}
\author{Y. B. Shen\textsuperscript{1,2}\thanks{shenyubing@sianp.ac.cn}
   , Q. Gu\textsuperscript{3}, Z. G. Jiang\textsuperscript{3}, D. Gu\textsuperscript{3}, Z. H. Zhu\textsuperscript{1,2}\\
   \textsuperscript{1}Shanghai Institute of Applied Physics, Chinese Academy of Sciences, Shanghai 201800, China\\
   \textsuperscript{2}University of Chinese Academy of Sciences, Beijing 100049, China\\
   \textsuperscript{3}Shanghai Advanced Research Institute, Chinese Academy of Sciences, Shanghai 201210, China}
\maketitle

\begin{abstract}
   With the high beam current in storage ring, it is necessary to consider
   the instability problem caused by the heavy beam loading effect. It has
   been demonstrated that direct RF feedback (DRFB), autolevel control
   loop (ALC) and phase-lock loop (PLL) in the main cavity can lessen
   the impact of the beam effect. This paper regarded the beam, main
   cavity, harmonic cavity and feedback loops as double harmonic cavity
   system, and extended the transfer functions in the Pedersen model to
   this system. Some quantitative evaluations of simulation results have
   been got and conclusions have been drawn. In the case of a passive
   harmonic cavity, the optimization strategy of the controller parameters
   in the pre-detuning , ALC and PLL, as well as the gain and phase shift
   of DRFB were discussed. Meanwhile, it also involved the impact of the
   harmonic cavity feedback loop on the system stability at the optimum
   stretching condition when an active harmonic cavity was present.
   The research results can be used as guidelines for beam operation
   with beam current increasing in the future.
\end{abstract}

\section{Introduction}
The higher harmonic cavity (HHC) has been proven to improve beam
life and suppress instabilities through Landau damping without
affecting energy diffusion and brightness\cite{ref1}\cite{ref2}.
However, in the fundamental mode, passing through the HHC can
destabilize the beam according to the Robinson criterion,
requiring detuning of the main cavity to maintain stability\cite{ref3}.
To study RF system instability, the influence of HHC cannot be ignored.
Techniques such as direct RF feedback (DRFB) can reduce heavy
beam loading and increase beam current limit by increasing cavity
bandwidth and reducing RF cavity shunt impedance\cite{ref4}.
Autolevel control loop (ALC), and phase lock loop (PLL) in
low-level RF (LLRF) systems stabilize cavity voltage while
affecting overall system stability. Robinson stability criterion
calculates the maximum beam current for a single cavity in the
storage ring\cite{ref5}, and Pedersen's feedback loop model
explains Robinson instability using beam and generator current modulations\cite{ref6}.
This paper adds HHC, DRFB, ALC, and PLL to Pedersen's model to
analyze loop instability and analyze the influence of loading angle,
HHC detuning, ALC and PLL controller parameters, and DRFB gain and
phase shift on the maximum beam current limit. Taking the Shanghai
Synchrotron Radiation Facility(SSRF) as an example, the effects
of various parameters on instability were analyzed and suggestions were proposed.

\section{Model description}
Taking passive harmonic double cavity system as an example,
the steady-state phasor diagram is shown in Fig.1.The
voltage ${{\tilde{V}}_{C}}$in the main cavity is determined
by beam current ${{\tilde{I}}_{B}}$, excitation source current
$\tilde{I}_{G}^{{}}$, DRFB current $\tilde{I}_{F}^{{}}$ and
cavity impedance. ${{\varphi }_{s}}$,${{\varphi }_{L}}$ and ${{\theta }_{L}}$
are the synchronization angle, detuning angle, and pre-detuning angle, respectively.
The total voltage ${{\tilde{V}}_{T}}$ is the vector sum of the
main cavity voltage and the passive HHC voltage ${{\tilde{V}}_{H}}$.
\begin{figure}[!htb]
   \centering
   \includegraphics*[width=.5\columnwidth]{THPA037_f1}
   \caption{Phasor diagram for the steady-state case,
      depicting the amplitude and phase of each voltage
      and current of the transmitter, cavity and beam.}
   \label{fig:paper_layout}
\end{figure}
\\
\hspace*{1em}$Y={{{I}_{B}}}/{{{I}_{0}}}\;$ is usually
used to characterize the severity of the beam loading effect, where
${{I}_{0}}$ is the projection of ${{I}_{T}}$ onto${{V}_{C}}$. Based
on this, the parameter $X={{{I}_{F}}}/{{{I}_{0}}}\;$
can be defined to characterize the gain of the feedback current,
while the phase can be represented by ${{\varphi }_{F}}$.
In the case of high-Q, the detuning angle ${{\varphi }_{H}}$
can be figured out to be about 90°, and the passive HHC voltage can be calculated from
${{V}_{H}}={{I}_{B}}\frac{{{r}_{p}}}{{{Q}_{p}}}\frac{{{f}_{hrf}}}{\Delta f}$
\cite{ref7}, where ${{f}_{hrf}}$
is n times of the RF frequency, ${\Delta f}$ is the detuning frequency.
The total cavity voltage can be determined by
${{V}_{T}}$ and ${{\theta }_{T}}$.
\begin{equation}\label{eq:label}
   \left\{ \begin{matrix}
      {{V}_{T}}=\sqrt{V_{C}^{2}+V_{H}^{2}-2{{V}_{C}}{{V}_{H}}\cos ({{\varphi }_{s}}-{{\varphi }_{H}})}                                                                            \\
      {{\theta }_{T}}=\arctan \left( -\frac{{{V}_{C}}\cos {{\varphi }_{s}}-{{V}_{H}}\cos {{\varphi }_{H}}}{{{V}_{C}}\sin {{\varphi }_{s}}-{{V}_{H}}\sin {{\varphi }_{H}}} \right) \\
   \end{matrix} \right.
\end{equation}
\hspace*{1em}The detuning angle of the main cavity can be
obtained by phasor diagram, which is equal to
\begin{footnotesize}
   \begin{equation}\label{eq:label}
      \tan {{\varphi }_{L}}=X\sin {{\varphi }_{F}}-Y\sin {{\varphi }_{s}}+\left( 1+Y\cos {{\varphi }_{s}}-X\cos {{\varphi }_{F}} \right)\tan {{\theta }_{L}}
   \end{equation}
\end{footnotesize}
\hspace*{1em}The Pedersen model of the passive harmonic double cavity
system can be deduced from the expression of the vector relationship
and the impedance of the cavity, as shown in Fig.2.
\begin{figure}[!htb]
   \centering
   \includegraphics*[width=1\columnwidth]{THPA037_f2}
   \caption{Expanded Pedersen model with passive HHC, ALC, PLL, and DRFB added.}
   \label{fig:paper_layout}
\end{figure}

Due to the slow response of tuner loop, only the ALC and PLL are
considered in the simulation\cite{ref8}, Note that this work
only includes static beam loading effects.
The transfer function that relates the modulation of current
excitation to the
modulation of cavity voltage signal in HHC is as follows:
\begin{equation}\label{eq:label}
   \left\{ \begin{matrix}
      G_{pa}^{BP}=\frac{\sigma _{p}^{{}}\tan {{\varphi }_{H}}s}{{{s}^{2}}+2\sigma _{p}^{{}}s+\sigma _{p}^{2}(1+{{\tan }^{2}}{{\varphi }_{H}})}                             \\
      G_{pp}^{BP}=\frac{\sigma _{p}^{2}(1+{{\tan }^{2}}{{\varphi }_{H}})+\sigma _{p}^{{}}s}{{{s}^{2}}+2\sigma _{p}^{{}}s+\sigma _{p}^{2}(1+{{\tan }^{2}}{{\varphi }_{H}})} \\
   \end{matrix} \right.
\end{equation}
\hspace*{1em}Where ${\sigma \text{=}{{{\omega }_{rf}}}/{\text{(}2{{Q}_{L}}\text{)}}}$ is the cavity damping factor,
${{\sigma }_{p}}$ is the cavity damping factor of passive HHC. According to the vector relationship
\begin{small}
   \begin{equation}\label{eq:label}
      \left\{ \begin{matrix}
         G_{pa}^{F}=\frac{{{I}_{F}}}{{{I}_{T}}}\left[ {{G}_{aa}}\sin \left( {{\varphi }_{F}}-{{\varphi }_{L}} \right)+{{G}_{pa}}\cos \left( {{\varphi }_{F}}-{{\varphi }_{L}} \right) \right]  \\
         G_{pp}^{F}=\frac{{{I}_{F}}}{{{I}_{T}}}\left[ {{G}_{ap}}\sin \left( {{\varphi }_{F}}-{{\varphi }_{L}} \right)+{{G}_{pp}}\cos \left( {{\varphi }_{F}}-{{\varphi }_{L}} \right) \right]  \\
         G_{pa}^{B}=\frac{{{I}_{B}}}{{{I}_{T}}}\left[ -{{G}_{aa}}\sin \left( {{\varphi }_{s}}-{{\varphi }_{L}} \right)-{{G}_{pa}}\cos \left( {{\varphi }_{s}}-{{\varphi }_{L}} \right) \right] \\
         G_{pp}^{B}=\frac{{{I}_{B}}}{{{I}_{T}}}\left[ -{{G}_{ap}}\sin \left( {{\varphi }_{s}}-{{\varphi }_{L}} \right)-{{G}_{pp}}\cos \left( {{\varphi }_{s}}-{{\varphi }_{L}} \right) \right] \\
      \end{matrix} \right.
   \end{equation}
\end{small}
\hspace*{1em}By the same token, we can get transfer functions such as ${G_{pa}^{G}}$,${G_{pp}^{G}}$,${G_{ap}^{F}}$,${G_{aa}^{F}}$.
${{B}_{s}}={\Omega _{s}^{2}}/{\left( {{s}^{2}}+{{\alpha }_{s}}\cdot s+\Omega _{s}^{2} \right)}\;$ is
the transfer function from the equivalent phase modulation of the total cavity voltage to the phase modulation of the beam current, where
$\Omega _s^{}$  is the longitudinal oscillation frequency\cite{ref9}.
Main cavity and harmonic cavity can both affect the equivalent phase of the total cavity voltage\cite{ref10}, and the weight of each component is
\begin{small}
   \begin{equation}\label{eq:label}
      \left\{ \begin{matrix}
         G_{ab}=[-\cos ({{\theta }_{T}}-{{\varphi }_{s}})-\sin ({{\theta }_{T}}-{{\varphi }_{s}})\tan {{\theta }_{T}}]\frac{{{V}_{C}}}{{{V}_{T}}}    \\
         G_{pb}=[-\sin ({{\theta }_{T}}-{{\varphi }_{s}})+\cos ({{\theta }_{T}}-{{\varphi }_{s}})\tan {{\theta }_{T}}]\frac{{{V}_{C}}}{{{V}_{T}}}    \\
         G_{ab}^{p}=[\cos ({{\theta }_{T}}-{{\varphi }_{H}})+\sin ({{\theta }_{T}}-{{\varphi }_{H}})\tan {{\theta }_{T}}]\frac{{{V}_{H}}}{{{V}_{T}}} \\
         G_{pb}^{p}=[\sin ({{\theta }_{T}}-{{\varphi }_{H}})-\cos ({{\theta }_{T}}-{{\varphi }_{H}})\tan {{\theta }_{T}}]\frac{{{V}_{H}}}{{{V}_{T}}} \\
      \end{matrix} \right.
   \end{equation}
\end{small}
\hspace*{1em}In DRFB loop, the amplifier needs to convert the voltage modulation signal into current modulation signal
${{G}_{f}}={X}/{{{R}_{L}}}$, where ${{R}_{L}}$ is the load shunt impedance in main cavity. The ALC and PLL controller is represented
as low pass filter that does not include the carrier frequency portion and removes the DC component\cite{ref11}. Furthermore,
Where ${{K}_{Ca}}$ and ${{K}_{Cp}}$ represent the gain, ${{C}_{a}}$ and ${{C}_{p}}$ represent the bandwidth.
\begin{equation}\label{eq:label}
   {{C}_{a,p}}=\frac{{{\omega }_{a,p}}}{s+{{\omega }_{a,p}}}
\end{equation}

\section{Influence of loop parameters on system performance}
Robinson instability calculates the maximum beam current of the bunch in a single cavity
, which occurs when ${{\tilde{V}}_{G}}$ and ${{\tilde{I}}_{B}}$ in the vector diagram are
in opposite phases. However, when additional loops are added, the coupling between the loops
makes it difficult to intuitively demonstrate instability analysis in the vector diagram.
In this case, New model can calculate the open-loop transfer function and draw the Nyquist
curve. When appropriate controller parameters are set, the poles of the system will not
fall in right half-plane. According to the Nyquist stability criterion, when the number
of crossings of the open-loop magnitude plot (positive frequency) with the left-hand
side of the (-1, 0j) point on the real axis is zero, there are no poles in the right
half-plane of the closed-loop system, then the system is stable.
Due to the negligible effect of the klystron's control function on beam dynamics, the loop
delay time T is approximately 1-2$\mu s$ \cite{ref13}. The zero-mode oscillation frequency
of SSRF is approximately 4.8kHz, and the delay function is approximately
${e^{ - sT}} \approx 1 - sT \approx 1$. Therefore, to simplify the model, delay is not considered.
\begin{table}[!hbt]
   \centering
   \caption{High-frequency parameters of SSRF\cite{ref12}\cite{ref13}}
   \begin{tabular}{lcc}
      \toprule
      %\textbf{Margin} & \textbf{A4 Paper}                      \\
      \midrule
      Energy                       & 3.5GeV     \\ %[3pt]
      RF frequency                 & 499.654MHz \\ %[3pt]
      Harmonic number              & 720        \\ %[3pt]
      Radiation loss               & 1.44MeV    \\
      Main RF voltage              & 4.5-5.4MV  \\
      r/Q of third harmonic cavity & 88         \\
      Harmonic cavity number       & 3          \\
      \bottomrule
   \end{tabular}
   \label{tab:margins}
\end{table}

The shunt impedance of the main high-frequency cavity is 28.5MΩ , the designed voltage
of the main cavity is 5.4 MV. The HHC voltage is approximately 1.8 MV under optimal
stretching conditions, the detuning frequency can be calculated as 22kHz.
\begin{figure}[!htb]
   \centering
   \includegraphics*[width=0.7\columnwidth]{THPA037_f3}
   \caption{The beam current is 300mA. DRFB is adjusted to achieve the critical
      stable state, where the curve passes through (-1, 0j), and this state is
      extended to double cavity system with different detunings, where the system remains in the critical stable state.}
   \label{fig:paper_layout}
\end{figure}
The stability can be determined by the gain margin, represented by SC for single
cavity and DC for double cavity, as seen in Fig.3 Nyquist plot. Critical stability
states are achieved by adjusting the gain and phase shift angle of DRFB as shown in Fig.4.
\begin{figure}[!htb]
   \centering
   \includegraphics*[width=0.7\columnwidth]{THPA037_f4}
   \caption{System gain margin versus phase shift angle, that is calculated with
      single cavity and double cavity, X=0-1.}
   \label{fig:paper_layout}
\end{figure}

Setting the phase shift angle in the range of -180° to -300° can make the system
more stable. In addition, adding HHC in the stable state will lead to a decrease in stability margin.
\begin{figure}[!htb]
   \centering
   \includegraphics*[width=0.7\columnwidth]{THPA037_f5}
   \caption{System gain margin versus Pre-detuning angle, that is calculated with HHC detuning frequency 18-26 kHz.}
   \label{fig:paper_layout}
\end{figure}

From Fig.5, it can be seen that the system is most unstable when the
pre-detuning angle is between 0-30°. In practice, due to the poor loop
control capability, there is often a deviation in the beam loading angle.
To prevent it from falling into the most unstable region, the loading
angle can be preset to a small negative value.

After considering ALC and PLL loops, Assume that the initial controller gain is
6 and bandwidth is 1 kHz. The effects of their gain and bandwidth on system stability are discussed separately\cite{ref14}.
\begin{figure}[!htb]
   \centering
   \includegraphics*[width=0.8\columnwidth]{THPA037_f6}
   \caption{Nyquist plot with gradually increasing amplitude loop gain, system from stable to unstable.}
   \label{fig:paper_layout}
\end{figure}

By analyzing the Nyquist plot as fig.6, it can be concluded that the stability of
various curves can be compared by the phase margin. When the ALC gain increases,
the phase margin decreases. However, simulation results have confirmed that variations
in PLL gain and controller bandwidth within a certain range do not affect stability margin.

After being converted to an active HHC, the optimal stretching state can be achieved
at different beam intensities. Adjusting the HHC transmitter coupling coefficient
can meet the optimal coupling, satisfying ${{\beta }_{op}}=1+{{{P}_{B}}}/{{{P}_{H}}}\;$\cite{ref15},
and its ALC and PLL controllers are consistent with the main cavity. However, it is found that the system is prone to enter an unstable state as shown in Fig.7.
\begin{figure}[!htb]
   \centering
   \includegraphics*[width=0.7\columnwidth]{THPA037_f7}
   \caption{Nyquist plot of the system. operation1: Reduce controller gain to 2 and bandwidth to 500 Hz; operation2: DRFB
      $X=1,{\varphi _F}=-260^\circ $; operation3: The coupling coefficient is reduced to one tenth of the optimal coupling.}
   \label{fig:paper_layout}
\end{figure}

Three solutions are proposed:
\begin{Itemize}
   \item  Reducing the gain and bandwidth of each controller, even if it may result in slower feedback control;
   \item  Using DRFB, but a high feedback gain may result in a decrease in the precision of cavity voltage control or even produce self-excited oscillation in the loop;
   \item  Decreasing the coupling coefficient of the transmitter of the HHC.
\end{Itemize}
\section{Conclusion}
This article proposes a novel mathematical method based on the Pedersen model
to analyze the stability of harmonic double cavity system for the first time.
The model utilizes control theory to provide a clear description of the effects
of each variable parameter on the stability of the system. Taking SSRF as an
example, it is discussed that the addition of a passive HHC does not affect
the maximum stable current, but it reduces the stability margin of the system
in the stable state. Optimization strategies for system stability are
presented by adjusting the parameters of pre-tuning angle, DRFB, ALC, and PLL.
Furthermore, the model is extended to an active HHC system, and it is found
that this system is prone to unstable states. Three upgrade proposals for
future active harmonic systems are proposed in this paper.

% only for "biblatex"
%
\ifboolexpr{bool{jacowbiblatex}}%
{\printbibliography}%
{%
   % "biblatex" is not used, go the "manual" way

   \begin{thebibliography}{99}   % Use for  10-99  references
      %\begin{thebibliography}{9} % Use for 1-9 references

      \bibitem{ref1}
      L. H. Chang, Ch. Wang, M. C. Lin, and M. S. Yeh, “Effects of the Passive Harmonic Cavity on the Beam Bunch”, in
      \textit{Proceedings of the 2005 Particle Accelerator Conference,},
      May 2005, pp. 3904-3906. doi: 10.1109/PAC.2005.1591663.

      \bibitem{ref2}
      R. A. Bosch and C. S. Hsue, “Suppression of longitudinal coupled-bunch instabilities by a passive higher harmonic cavity”, in
      \textit{Proceedings of International Conference on Particle Accelerators,},
      May 1993, pp. 3369-3371 vol.5. doi: 10.1109/PAC.1993.309653.

      \bibitem{ref3}
      K. Ng, “Passive landau cavity for the LNLS light source electron ring”, 2000.

      \bibitem{ref4}
      M. G. Minty and R. H. Siemann, “Heavy beam loading in storage ring radio frequency systems”,
      \textit{Nuclear Instruments and Methods in Physics Research Section A: Accelerators, Spectrometers, Detectors and Associated Equipment,},
      vol. 376, no. 3, pp. 301-318, Jul. 1996, doi: 10.1016/0168-9002(96)00180-5.

      \bibitem{ref5}
      K. W. Robinson, “Stability of beam in radiofrequency system”, Cambridge Electron accelerator, Mass., CEAL-1010, Feb. 1964. doi: 10.2172/4075988.

      \bibitem{ref6}
      F. Pedersen, “Beam Loading Effects in the CERN PS Booster”,
      \textit{IEEE Transactions on Nuclear Science,},
      vol. 22, no. 3, pp. 1906-1909, Jun. 1975, doi: 10.1109/TNS.1975.4328024.

      \bibitem{ref7}
      P. Marchand, “Possible upgrading of the SLS RF system for improving the beam lifetime”,
      \textit{Proceedings of the 1999 Particle Accelerator Conference (Cat. No.99CH36366),},
      pp. 989-991 vol.2, 1999, doi: 10.1109/PAC.1999.795424.

      \bibitem{ref8}
      K. Akai, “Stability analysis of rf accelerating mode with feedback loops under heavy beam loading in SuperKEKB“,
      \textit{Phys. Rev. Accel. Beams,},
      vol. 25, no. 10, 2022, doi: 10.1103/PhysRevAccelBeams.25.102002.

      \bibitem{ref9}
      S. Y. Zhang and W. T. Weng, “Error analysis of acceleration control loops of a synchrotron”,
      \textit{AIP Conference Proceedings,},
      vol. 255, no. 1, pp. 181-196, May 1992, doi: 10.1063/1.42317.

      \bibitem{ref10}
      S. Belomestnykh, R. Kaplan, J. Reilly, and V. Veshcherevich, “Instability of the RF Control Loop in the Presence of a High-Q Passive Superconducting Cavity“, in
      \textit{Proceedings of the 2005 Particle Accelerator Conference,},
      May 2005, pp. 2553-2555. doi: 10.1109/PAC.2005.1591178.

      \bibitem{ref11}
      Y.-Y. Xia et al., “Transfer function measurement for the SSRF SRF system”,
      \textit{NUCL SCI TECH,},
      vol. 30, no. 6, p. 101, May 2019, doi: 10.1007/s41365-019-0612-4.

      \bibitem{ref12}
      X.-Y. Pu et al., “Frequency sensitivity of the passive third harmonic superconducting cavity for SSRF”,
      \textit{NUCL SCI TECH, },
      vol. 31, no. 3, p. 31, Feb. 2020, doi: 10.1007/s41365-020-0732-x.

      \bibitem{ref13}
      T. Phimsen et al., “Improving Touschek lifetime and synchrotron frequency spread by passive harmonic cavity in the storage ring of SSRF”,
      \textit{NUCL SCI TECH,},
      vol. 28, no. 8, p. 108, Jun. 2017, doi: 10.1007/s41365-017-0259-y.

      \bibitem{ref14}
      Z.-K. Liu et al., “Modeling the interaction of a heavily beam loaded SRF cavity with its low-level RF feedback loops”,
      \textit{Nuclear Instruments and Methods in Physics Research Section A: Accelerators, Spectrometers, Detectors and Associated Equipment,},
      vol. 894, pp. 57-71, Jun. 2018, doi: 10.1016/j.nima.2018.03.046.

      \bibitem{ref15}
      A. Chao, “Physics Of Collective Beam Instabilities In High Energy Accelerators”, Jan. 1993

   \end{thebibliography}
} % end \ifboolexpr
%
% for use as JACoW template the inclusion of the ANNEX parts have been commented out
% to generate the complete documentation please remove the "%" of the next two commands
% 
%===\newpage

%===\include{annexes-Letter}

\end{document}